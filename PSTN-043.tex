\documentclass[modern]{aastex62}

% lsstdoc documentation: https://lsst-texmf.lsst.io/lsstdoc.html
\input{meta}

% Package imports go here.

% Local commands go here.



\newcommand{\docRef}{PSTN-043}
\newcommand{\docUpstreamLocation}{\url{https://github.com/lsst-pst/pstn-043}}


\begin{document}
\input{authors}
\date{\today}
\title{Scheduling for the Rubin Observatory Legacy Survey of Space and Time}
\hypersetup{pdftitle={\@title}, pdfauthor={\@author}, pdfkeywords={\@keywords}}


\begin{abstract}
 
As the Commissioning Execution Plan (LSE-390) says, "The project team shall
deliver all reports documenting the as-built hardware and software including:
drawings, source code, modifications, compliance exceptions, and recommendations
for improvement." As a first step towards the delivery of documents that will describe the system at the
end of construction, we are assembling teams for producing of the order 40 papers
that eventually will be submitted to relevant professional journals. The immediate goal is to accomplish
all the writing that can be done without data analysis before the data
taking begins, and the team becomes much more busy and stressed.

This document provides the template for these papers.
\end{abstract}



% put the outline into 'body' for now, split into sections later maybe
\section{Introduction}

\begin{itemize}
\item Overview of the general requirements for the LSST scheduling 'brain'  (autonomous, reconfigurable, responsive to conditions, etc)
\item What is our general approach for the scheduler algorithms (vs. other approaches) and why was the Markov-Decision style Feature Based Scheduler chosen?
\item Outline sections in remainder of paper
\end{itemize} 

\section{Survey Strategy Goals}
Lynne

\begin{itemize}
\item SRD and four primary science themes - how do they drive scheduling?
\item These primary goals require 80-90\% of the survey time, so what is the starting point for deciding on the rest of the time?
\item Outline community driven process for determining survey strategy
\end{itemize}

\section{Feature Based Scheduler}
Peter

\begin{itemize}
\item Basic outline of scheduler code - track 'features' and calculate rewards/goals using basis functions, combine basis functions to choose pointing for a `survey'
\item Combine various surveys, including scripted surveys that allow us to add more rigid requirements
\item Different kinds of surveys, including greedy / blob / scripted
\item When do you split between scripted survey and general survey, when between blob and greedy, why do we have surveys for each filter
\item Within the greedy and blob surveys, what are the basis functions we are currently using? 
\item How would you modify the scheduler to add more?
\item How do the scripted surveys work? When do they trigger? How would you add an optimizer?  
\end{itemize}

\section{Creating Simulated Pointing Histories}
Peter/Tiago/Lynne

\begin{itemize}
\item Running the FBS brain within the Operations Simulation environment
\item What are the telescope model, and weather models? Where do they get their inputs? How does this translate into operations? (reference ptsn-007)
\end{itemize}

\section{Evaluating the Long-Term Survey Strategy}
Lynne/Peter

\begin{itemize}
\item Range of survey strategies studied in the simulations 
\item Metrics used to evaluate broad science themes 
\item Metrics used to evaluate all science themes, including options for smaller level optimization of the survey strategy
\item Refer to report to the SCOC
\end{itemize}

\section{Evaluating Survey Progress}
Lynne/Peter

\begin{itemize}
\item Evaluate survey progress on short term (daily/ weekly), quarterly and twice-yearly timescales
\item Metrics for evaluating survey progress on short terms and comparing to expected performance
\item Process for adjusting scheduler to account for short-term variations if needed (when will it be needed?)
\item Metrics for evaluating survey progress on quarterly/yearly timescales and comparing to expected performance
\item Process and types of adjustments possible for quarterly/yearly timescales, how does it involve the SCOC?
\end{itemize}

\section{Conclusions}

review and add thoughts on future changes


\appendix
% Remove this when you start your paper
%
{\bf Initial paper list added here for reference.}

``Editor'' is a responsible team leader but not necessarily the person who will do most of
the required work, or who will eventually become the first author. Both issues will be
handled by individual teams.

\begin{verbatim}

domain: Telescope & Site
editor: Jeff Barr
title: Overview of the LSST Telescope

domain: Telescope & Site
editor: Sandrine Thomas
title: Performance of the LSST Telescope

domain: Telescope & Site
editor: Lynne Jones
title: The LSST Scheduler Overview and Performance

domain: Telescope & Site
editor: Bo Xin
title: Performance of the LSST Active Optics System

domain: Telescope & Site
editor: Tiago Ribeiro
title: LSST Observing System Software Architecture

domain: Camera
editor: Justin Wolfe
title: LSST Camera Optics

domain: Camera
editor: Chris Stubbs
title: LSST Camera Rafts

domain: Camera
editor: Steve Ritz
title: LSST Camera Cryostat

domain: Camera
editor: Ralph Schindler
title: LSST Camera Refrigeration

domain: Camera
editor: Steve Ritz
title: LSST Camera Body and Mechanisms

domain: Camera
editor: Mark Huffer and Tony Johnson
title: LSST Camera Control System and DAQ

domain: Camera
editor: Tim Bond and Aaron Rodman
title: LSST Camera Integration and Tests

domain: Data Management
editor: Leanne Guy
title: Overview of LSST Data Management

domain: Data Management
editor: Michelle Butler
title: LSST Data Facility

domain: Data Management
editor: Tim Jenness
title: LSST Data Management Software System

domain: Data Management
editor: Jim Bosch
title: LSST Data Release Processing

domain: Data Management
editor: Eric Bellm
title: LSST Prompt Data Products

domain: Data Management
editor: Gregory Dubois-Felsmann
title: LSST Science Platform

domain: Data Management
editor: Simon Krughoff
title: LSST Data Management Quality Assurance and Reliability Engineering

domain: Data Management
editor: Leanne Guy (with likely delegation to new DM V&V Scientist)
title: LSST Data Management System Verification and Validation

domain: Data Management
editor: Mario Juric
title: LSST Moving Object Processing

domain: Data Management
editor: Robert Lupton
title: LSST Calibration Strategy and Pipelines

domain: Calibration
editor: Patrick Ingraham
title:  Performance of the LSST Calibration Systems

domain: Calibration
editor: Patrick Ingraham
title: Atmospheric Properties with the LSST Auxiliary Telescope

domain: EPO
editor: Amanda Bauer
title: Overview of LSST Education and Public Outreach

domain: EPO
editor: Ardis Herrold
title: LSST Formal Education Program

domain: EPO
editor: Amanda Bauer
title: LSST EPO: The User Feedback

domain: Commissioning
editor: Chuck Claver
title: LSST Observatory System Operations Readiness Report

domain: Commissioning
editor: Bo Xin
title: Performance of Delivered LSST System

domain: Commissioning
editor: Chuck Claver
title: Active Optics Performance with LSST Commissiong Camera

domain: Commissioning
editor: Chuck Claver
title: LSST Active Optics Performance with the LSST Science Camera

domain: Commissioning
editor: Brian Stalder
title: Integration, Test and Commissioning Results from LSST Commissiong Camera

domain: Commissioning
editor: Kevin Reil
title: LSST Camera Instrumental Signature Characterization, Calibration and Removal

domain: Commissioning
editor: Patrick Hascal
title: Installation and Performance of the LSST Camera Refrigeration System

domain: Commissioning
editor: Andy Connolly
title: Science Validation of LSST Alert Processing

domain: Commissioning
editor: Keith Bechtol
title: Science Validation of LSST Data Release Processing

domain: Commissioning
editor: Michael Reuter
title: Tracking of LSST System Performance with Continuous Integration Methods

domain: Commissioning
editor: Chuck Claver
title: The LSST Science Platform as a Commissioning Tool

domain: Commissioning
editor: Chuck Claver
title: Commissioning Science Data Quality Analysis Tools, Methods and Procedures

domain: Commissioning
editor: Lynne Jones
title: Performance Verification of the LSST Survey Scheduler


\end{verbatim}

% Include all the relevant bib files.
% https://lsst-texmf.lsst.io/lsstdoc.html#bibliographies
\section{References} \label{sec:bib}
\bibliographystyle{yahapj}
\bibliography{local,lsst,lsst-dm,refs_ads,refs,books}

% Make sure lsst-texmf/bin/generateAcronyms.py is in your path
\section{Acronyms} \label{sec:acronyms}
\input{acronyms.tex}

\end{document}
