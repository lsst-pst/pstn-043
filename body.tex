\section{Introduction}

\begin{itemize}
\item Overview of the general requirements for the LSST scheduling 'brain'  (autonomous, reconfigurable, responsive to conditions, etc)
\item What is our general approach for the scheduler algorithms (vs. other approaches) and why was the Markov-Decision style Feature Based Scheduler chosen?
\item Outline sections in remainder of paper
\end{itemize} 

\section{Survey Strategy Goals}
Lynne

\begin{itemize}
\item SRD and four primary science themes - how do they drive scheduling?
\item These primary goals require 80-90\% of the survey time, so what is the starting point for deciding on the rest of the time?
\item Outline community driven process for determining survey strategy
\end{itemize}

\section{Feature Based Scheduler}
Peter

\begin{itemize}
\item Basic outline of scheduler code - track 'features' and calculate rewards/goals using basis functions, combine basis functions to choose pointing for a `survey'
\item Combine various surveys, including scripted surveys that allow us to add more rigid requirements
\item Different kinds of surveys, including greedy / blob / scripted
\item When do you split between scripted survey and general survey, when between blob and greedy, why do we have surveys for each filter
\item Within the greedy and blob surveys, what are the basis functions we are currently using? 
\item How would you modify the scheduler to add more?
\item How do the scripted surveys work? When do they trigger? How would you add an optimizer?  
\end{itemize}

\section{Creating Simulated Pointing Histories}
Peter/Tiago/Lynne

\begin{itemize}
\item Running the FBS brain within the Operations Simulation environment
\item What are the telescope model, and weather models? Where do they get their inputs? How does this translate into operations? (reference ptsn-007)
\end{itemize}

\section{Evaluating the Long-Term Survey Strategy}
Lynne/Peter

\begin{itemize}
\item Range of survey strategies studied in the simulations 
\item Metrics used to evaluate broad science themes 
\item Metrics used to evaluate all science themes, including options for smaller level optimization of the survey strategy
\item Refer to report to the SCOC
\end{itemize}

\section{Evaluating Survey Progress}
Lynne/Peter

\begin{itemize}
\item Evaluate survey progress on short term (daily/ weekly), quarterly and twice-yearly timescales
\item Metrics for evaluating survey progress on short terms and comparing to expected performance
\item Process for adjusting scheduler to account for short-term variations if needed (when will it be needed?)
\item Metrics for evaluating survey progress on quarterly/yearly timescales and comparing to expected performance
\item Process and types of adjustments possible for quarterly/yearly timescales, how does it involve the SCOC?
\end{itemize}

\section{Conclusions}

review and add thoughts on future changes
